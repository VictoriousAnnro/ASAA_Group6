\documentclass[conference]{IEEEtran}
\IEEEoverridecommandlockouts
% The preceding line is only needed to identify funding in the first footnote. If that is unneeded, please comment it out.
\usepackage{cite}
\usepackage{amsmath,amssymb,amsfonts}
\usepackage{algorithmic}
\usepackage{graphicx}
\usepackage{textcomp}
\usepackage{xcolor}
\usepackage{multirow}
\usepackage{rotating}
\usepackage{mdframed}
\usepackage{hyperref}
\usepackage{tikz}
\usepackage{makecell}
\usepackage{tcolorbox}
\usepackage{amsthm}
%\usepackage[english]{babel}
\usepackage{pifont} % checkmarks
%\theoremstyle{definition}
%\newtheorem{definition}{Definition}[section]


\usepackage{listings}
\lstset
{ 
    basicstyle=\footnotesize,
    numbers=left,
    stepnumber=1,
    xleftmargin=5.0ex,
}

%SCJ
\usepackage{subcaption}
\usepackage{array, multirow}
\usepackage{enumitem}

\def\BibTeX{{\rm B\kern-.05em{\sc i\kern-.025em b}\kern-.08em
    T\kern-.1667em\lower.7ex\hbox{E}\kern-.125emX}}
\begin{document}

%\IEEEpubid{978-1-6654-8356-8/22/\$31.00 ©2022 IEEE}
% @Sune:
% Found this suggestion: https://site.ieee.org/compel2018/ieee-copyright-notice/
% I have added it - you can see if it fulfills the requirements

%\IEEEoverridecommandlockouts
%\IEEEpubid{\makebox[\columnwidth]{978-1-6654-8356-8/22/\$31.00 ©2022 IEEE %\hfill} \hspace{\columnsep}\makebox[\columnwidth]{ }}
                                 %978-1-6654-8356-8/22/$31.00 ©2022 IEEE
% copyright notice added:
%\makeatletter
%\setlength{\footskip}{20pt} 
%\def\ps@IEEEtitlepagestyle{%
%  \def\@oddfoot{\mycopyrightnotice}%
%  \def\@evenfoot{}%
%}
%\def\mycopyrightnotice{%
%  {\footnotesize 978-1-6654-8356-8/22/\$31.00 ©2022 IEEE\hfill}% <--- Change here
%  \gdef\mycopyrightnotice{}% just in case
%}
      
\title{Group Report Template\\
}

\author{
    \IEEEauthorblockN{
        Anne-Marie Rommerdahl\IEEEauthorrefmark{1},
        Jannatul Ferdous\IEEEauthorrefmark{1},
        Umma Soneyatul Jannat\IEEEauthorrefmark{1},
        Student 4\IEEEauthorrefmark{1},
        
    }
    \IEEEauthorblockA{
        University of Southern Denmark, SDU Software Engineering, Odense, Denmark \\
        Email: \IEEEauthorrefmark{1} \textnormal{\{anrom,umjan25,student4,student5\}}@mmmi.sdu.dk
    }
}

%%%%

%\author{\IEEEauthorblockN{1\textsuperscript{st} Blinded for review}
%\IEEEauthorblockA{\textit{Blinded for review} \\
%\textit{Blinded for review}\\
%Blinded for review \\
%Blinded for review}
%\and
%\IEEEauthorblockN{2\textsuperscript{nd} Blinded for review}
%\IEEEauthorblockA{\textit{Blinded for review} \\
%\textit{Blinded for review}\\
%Blinded for review \\
%Blinded for review}
%\and
%\IEEEauthorblockN{3\textsuperscript{nd} Blinded for review}
%\IEEEauthorblockA{\textit{Blinded for review} \\
%\textit{Blinded for review}\\
%Blinded for review \\
%Blinded for review}
%}

%%%%
%\IEEEauthorblockN{2\textsuperscript{nd} Given Name Surname}
%\IEEEauthorblockA{\textit{dept. name of organization (of Aff.)} \\
%\textit{name of organization (of Aff.)}\\
%City, Country \\
%email address or ORCID}

\maketitle
\IEEEpubidadjcol
\begin{abstract}
%%%%%%%%%%%%%%%%%% Max 970 signs without space %%%%%%%%%%%%%%%%%%
% Intro

% Gap
    
% Aim 

% Method

% Results 

\end{abstract}

\begin{IEEEkeywords}
Keyword1, Keyword2, Keyword3, Keyword4, Keyword5
\end{IEEEkeywords}

\section{Introduction and Motivation}
%Introduction and motivate the problem

The structure of the paper is as follows. 
Section \ref{sec:problem} outlines the research question and the research approach. 
%to analyze the research question and evaluate our results.
Section \ref{sec:related_work} describes similar work in the field and how our contribution fits the field.
Section \ref{sec:use_case} presents a production reconfiguration use case.
The use case serves as input to specify a reconfigurability QA requirement in Section \ref{sec:qas}.
Section \ref{sec:middleware_architecture} introduces the proposed reconfigurable middleware software architecture design.
Section \ref{sec:evaluation} evaluates the proposed middleware on realistic equipment in the I4.0 lab and analyzes the results against the stated QA requirement.   

\section{Problem and Approach}

\label{sec:problem}
% What is the problem* to be solved with the reliable architecture you build, and how will the problem be addressed. *The stated problem leads to the stated research question.
\emph{Problem.}
%AM
In todays society, there is a heavy focus on the automation and digitalization of complex systems. The complex systems are used in many areas, such as industry, healthcare, public transport and so on. These systems are described as complex, because of the many parts involved, the intricate ways in which those parts communicate, and the overall behavior of the system - how do all these parts work together to complete the systems' goals? To ensure the correctness and reliability of such systems, it is of outmost importance that thought is put into the design. After all, failure in a system can lead to negative consequences ranging from minor inconviniences to death[SOURCE].\\
The Industry 4.0 production (I4.0) domain [... need to define]. In this domain there exist many complex production systems, one of which is the production of cars. Cars are sold not only for their functionality as a vehicle of transport, but often offer the customers a range of options for customization. These range from functional (such as the engine) to cosmetic (such as the color). Building a system that can handle customer orders and organize production of the placed orders is no small feat. It is one thing to design an architecture for such a system, but how does one ensure that it fulfills the requirements? Changes to the system during later parts of the development process are much more costly than changes made earlier in the process. Therefore it is important to ensure that the architecture is designed well, before implementation starts. From this, we've created these research questions:
\emph{Research questions:}
\begin{enumerate}
    \item  How to design software architecture that adopts defined architectural requirements?
    \item How to evaluate the architectural requirements?
    \item What trade-offs arise from the choosen technologies?
    \item ...smth how effective is our architecture in ensuring the desired quality attributes?** PROLLY DELETE THIS
\end{enumerate}

\emph{Approach.}
The following steps are taken to answer this paper's research questions:
\begin{enumerate}
    \item DO LATER
\end{enumerate}


\section{Related work}
\label{sec:related_work}
% ... Should review the state of the art consisting of 8-10 papers and should contextualize how this study provides new knowledge to the field. 
This Section addresses existing contributions by examining xxx in the I4.0 domain. 
In total, x papers are investigated. 

In \cite{Wan2019Reconfigurable}, experiences are elaborated on a three-layer architecture of a reconfigurable smart factory for drug packing in healthcare I4.0. 

The paper \cite{Yazen2010Ontology} proposes an ontology agent-based architecture for inferring  new configurations to adapt to changes in manufacturing requirements and/or environment.

In \cite{Leitao2016Specification,Angione2017Integration} an architecture for a reconfigurable production system is specified.
Two objectives for reconfiguration and how they can be reached are described.

Several papers \cite{Koren1999Reconfigurable,Koren2010Design,Bortolini2018Reconfigurable} describe reconfigurable manufacturing systems that are cost-effective and responsive to market changes.

All contributions provide valuable knowledge about reconfiguration but lack a study of the software architecture perspective that specifies a quantifiable reconfigurability architectural requirement, a software architecture that adopts the architectural requirements, and evaluates the architectural requirement. 


\section{Use Case}
\label{sec:use_case}
% From lecture 12 Summary-keypoint-exam topics sample: 
% Unfold the problem with a use case and describe what the use case is about
% So i dont think we need all our use cases. Just one or a few relevant ones
This Section introduces the use case of a customer placing an order on the company's website. This use case covers the sequence of actions starting from the customer accessing the website, to the moment when the order is saved and marked as 'ready' for production in the system.
\subsection{Customer places an order}
Actors: Customer, Website, Production Scheduler
Preconditions: Available cars and customization options have been defined in database.
Steps:
\begin{itemize}
    \item Customer accesses company website
    \item Customer chooses preferred car
    \item Website fetches available customization options for chosen car
    \item Customer chooses preferred customizations
    \item Customer places the order
    \item Website sends order details to the Scheduler
    \item Production Scheduler recieves order details from website
    \item Production Scheduler breaks down the order into a JSON production recipe, which is then stored in the database
    \item Production Scheduler stores the order in the database, marked as 'ready'
\end{itemize}
Postconditions: A customer order is stored and marked as 'ready' for production.

\section{Quality Attribute Scenario}
This section contains the quality attributes for the architecture and also contains one or more Quality Attribute Scenarios, which illustrate the system's scenario.
\subsection{Performance}
\label{sec:qas}

The performance of the system starts from the moment a customer has placed an order on the website, and the performance depends on the website, Non/Non-relational DB, Kafka Message bus, and Production Scheduler. Communication between these systems must take place within ten units.\\

\textbf{Scenarios:} Two scenarios have been made for Performance.\\

The first scenario on 1 shows that when a customer places an order through the car-selling website, it will publish to Kafka that an order has been created. After this event is done, the production scheduler fetches the message and starts generating the JSON production recipe, and marks it as ready as soon as it is stored in the database. This entire event chain will be completed in 10 time units. The customer's order submission to Kafka to get the message; this process must complete within four milliseconds. This scenario ensures that distributed communication does not cause significant latency and captures the system's performance requirement for the major workflow.

\begin{figure}[h]
    \centering
    \includegraphics[width=0.8\linewidth]{quality attribute diagram 1.drawio.png} % adjust filename
    \caption{Scenario 1 for the Performance}
    \label{quality attribute diagram 1.drawio}
\end{figure}

The second scenario on 2 shows that after the Kafka message bus gets the order from the website, the scheduler must start immediately to prevent delays. In normal load conditions, as soon as the message is available in Kafka, the scheduler must start generating the recipe within three milliseconds. This prevents performance bottlenecks in the message queue and also ensures that the event-driven architecture facilitates quick transitions from order creation to recipe development.

\begin{figure}[h]
    \centering
    \includegraphics[width=0.8\linewidth]{quality attribute diagram 2.drawio.png} % adjust filename
    \caption{Scenario 2 for the Performance}
    \label{quality attribute diagram 2.drawio}
\end{figure}

In this subsection, the tactics that have been used to fulfill the Quality Attribute Scenarios are listed.\\
To ensure the health of the connection from the systems, the most fundamental tactic is event-based processing, which separates the Website from the Scheduler and guarantees that user-facing operations do not hinder generating the JSON production recipe or database writing. It also covers the bound execution times tactics, as after the customer places the order, the message bus will receive the message within three milliseconds, and the Production Scheduler will generate the recipe within three milliseconds. To reduce overall latency, the system additionally depends on concurrency by simultaneously allowing the Scheduler to process messages, create recipes, and communicate with the database. By doing that, it avoids bottlenecks, and concurrent processing guarantees that the Scheduler can process incoming Kafka events.

\section{Design and Analysis Modelling}
\label{sec:design_and_analysis_modelling}
This section will outline the design patterns suitable for the project.\\

 \subsubsection{Apache Kafka:} To establish communication between the website and the production scheduler, Apache Kafka will be used. The architecture requires a message bus for the subsystems to communicate with each other, and the system does not require a third-party broker or messages to be organized by "Topics," which is why Kafka is chosen instead of MQTT. Kafka decouples data streams and systems. The website sends data into Kafka, and the production scheduler queries Kafka for the data. It scales very well and won't require integration with every system.
 
 \subsection{ Systems and subsystems}
 \begin{itemize}
    \item{1.Website}
 \begin{itemize}
        \item Backend
      \begin{itemize}
        \item Programming language: \textbf{ Django}
        \begin{itemize}
            \item Comprehensibility of the language makes it easier and faster to write understandable code, and easier to comply with the system. 
            \item It has enough libraries to automate testing, as it can easily conduct testing. Also, it offers a platform that allows automation based on non-browser functionality. 
        \end{itemize}
        \item Database: SQLite
    \end{itemize}
\end{itemize}
\begin{itemize}
        \item Fronted 
      \begin{itemize}
        \item Programming language: \textbf{HTML,CSS,JavaScript }
        \begin{itemize}
            \item Runs anywhere, no installation needed.
            \item UI updates instantly for all users.
        \end{itemize}
    \end{itemize}
\end{itemize}
\end{itemize}



Design and analysis modelling.
% Description of the overall architecture designs
% Argue for tactics used to archieve the QASes
% Discuss the trade-offs
%AM
% Use cases -> RQs -> feature model -> architectural model/SM?
%feature model is optional. However, it should be xplained what/which particular features (characteristics) are selected and the reason of choise.
%describe the design and argue for the design decision and how it meets the QASes. Part of the design decision must specify which tactics/patterns are used (provide arguments) and the trade-offs. 
%analyse the design and illustrate the system architecture model, encompassing both structure and behavior, using formal reasoning

\section{Formal verification and validation}
\label{sec:formal_v_and_v}
Formal verification and validation of system(s).


\section{Evaluation}
\label{sec:evaluation}

% Empirical evaluation
This Section describes the evaluation of the proposed design.
Section \ref{sec:design} introduces the design of the experiment to evaluate the system. 
Section \ref{sec:measurements} identifies the measurements in the system for the experiment.
Section \ref{sec:pilot_test} describes the pilot test used to compute the number of replication in the actual evaluation. 
Section \ref{sec:analysis} presents the analysis of the results from the experiment. 

\subsection{Experiment design}
\label{sec:design}

\subsection{Measurements}
\label{sec:measurements}

\subsection{Pilot test}
\label{sec:pilot_test}

\subsection{Analysis}
\label{sec:analysis}


\section{Conclusion}
\label{sec:conclusion}
Conclusion of the report, discussion and relevant future work.

\subsection{Discussion}
\label{sec:Discussion}

\subsection{Future work}
\label{sec:future_work}

\bibliographystyle{IEEEtran}
\bibliography{references}
\vspace{12pt}

\newpage
\section*{Contributions}
\begin{tabular}{l|p{0.6\linewidth}}
Name & Contribution\\
Anne-Marie & Scheduler, Use cases\\
\end{tabular}
\end{document}
