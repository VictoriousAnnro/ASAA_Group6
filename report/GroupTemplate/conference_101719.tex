\documentclass[conference]{IEEEtran}
\IEEEoverridecommandlockouts
% The preceding line is only needed to identify funding in the first footnote. If that is unneeded, please comment it out.
\usepackage{cite}
\usepackage{amsmath,amssymb,amsfonts}
\usepackage{algorithmic}
\usepackage{graphicx}
\usepackage{textcomp}
\usepackage{xcolor}
\usepackage{multirow}
\usepackage{rotating}
\usepackage{mdframed}
\usepackage{hyperref}
\usepackage{tikz}
\usepackage{makecell}
\usepackage{tcolorbox}
\usepackage{amsthm}
\usepackage{array}
%\usepackage[english]{babel}
\usepackage{pifont} % checkmarks
%\theoremstyle{definition}
%\newtheorem{definition}{Definition}[section]


\usepackage{listings}
\lstset
{ 
    basicstyle=\footnotesize,
    numbers=left,
    stepnumber=1,
    xleftmargin=5.0ex,
}

%SCJ
\usepackage{subcaption}
\usepackage{array, multirow}
\usepackage{enumitem}

\def\BibTeX{{\rm B\kern-.05em{\sc i\kern-.025em b}\kern-.08em
    T\kern-.1667em\lower.7ex\hbox{E}\kern-.125emX}}
\begin{document}

%\IEEEpubid{978-1-6654-8356-8/22/\$31.00 ©2022 IEEE}
% @Sune:
% Found this suggestion: https://site.ieee.org/compel2018/ieee-copyright-notice/
% I have added it - you can see if it fulfills the requirements

%\IEEEoverridecommandlockouts
%\IEEEpubid{\makebox[\columnwidth]{978-1-6654-8356-8/22/\$31.00 ©2022 IEEE %\hfill} \hspace{\columnsep}\makebox[\columnwidth]{ }}
                                 %978-1-6654-8356-8/22/$31.00 ©2022 IEEE
% copyright notice added:
%\makeatletter
%\setlength{\footskip}{20pt} 
%\def\ps@IEEEtitlepagestyle{%
%  \def\@oddfoot{\mycopyrightnotice}%
%  \def\@evenfoot{}%
%}
%\def\mycopyrightnotice{%
%  {\footnotesize 978-1-6654-8356-8/22/\$31.00 ©2022 IEEE\hfill}% <--- Change here
%  \gdef\mycopyrightnotice{}% just in case
%}
      
\title{Group Report Template\\
}

\author{
    \IEEEauthorblockN{
        Anne-Marie Rommerdahl\IEEEauthorrefmark{1},
        Jannatul Ferdous\IEEEauthorrefmark{2},
        Umma Soneyatul Jannat\IEEEauthorrefmark{3},
        Student 4\IEEEauthorrefmark{4}
    }
    \IEEEauthorblockA{
        University of Southern Denmark, SDU Software Engineering, Odense, Denmark \\
        Email: \IEEEauthorrefmark{1} \textnormal{\{anrom,umjan25,student4,student5\}}@mmmi.sdu.dk
    }
}

%%%%

%\author{\IEEEauthorblockN{1\textsuperscript{st} Blinded for review}
%\IEEEauthorblockA{\textit{Blinded for review} \\
%\textit{Blinded for review}\\
%Blinded for review \\
%Blinded for review}
%\and
%\IEEEauthorblockN{2\textsuperscript{nd} Blinded for review}
%\IEEEauthorblockA{\textit{Blinded for review} \\
%\textit{Blinded for review}\\
%Blinded for review \\
%Blinded for review}
%\and
%\IEEEauthorblockN{3\textsuperscript{nd} Blinded for review}
%\IEEEauthorblockA{\textit{Blinded for review} \\
%\textit{Blinded for review}\\
%Blinded for review \\
%Blinded for review}
%}

%%%%
%\IEEEauthorblockN{2\textsuperscript{nd} Given Name Surname}
%\IEEEauthorblockA{\textit{dept. name of organization (of Aff.)} \\
%\textit{name of organization (of Aff.)}\\
%City, Country \\
%email address or ORCID}

\maketitle
\IEEEpubidadjcol
\begin{abstract}
%%%%%%%%%%%%%%%%%% Max 970 signs without space %%%%%%%%%%%%%%%%%%
% Intro

% Gap
    
% Aim 

% Method

% Results 

\end{abstract}

\begin{IEEEkeywords}
Keyword1, Keyword2, Keyword3, Keyword4, Keyword5
\end{IEEEkeywords}

\section{Introduction and Motivation}
%Introduction and motivate the problem

The structure of the paper is as follows. 
Section \ref{sec:problem} outlines the research question and the research approach to analyze the research question and evaluate our results.
Section \ref{sec:related_work} describes similar work in the field and how our contribution fits the field.
Section \ref{sec:use_case} presents a production reconfiguration use case.
The use case serves as input to specify a reconfigurability QA requirement in Section \ref{sec:qas}.
Section \ref{sec:middleware_architecture} introduces the proposed reconfigurable middleware software architecture design.
Section \ref{sec:evaluation} evaluates the proposed middleware on realistic equipment in the I4.0 lab and analyzes the results against the stated QA requirement.   

\section{Problem and Approach}
\label{sec:problem}
% What is the problem* to be solved with the reliable architecture you build, and how will the problem be addressed. *The stated problem leads to the stated research question.
\emph{Problem.}
In todays society, there is a heavy focus on the automation and digitalization of complex systems. The complex systems are used in many areas, such as industry, healthcare, public transport and so on. These systems are described as complex, because of the many parts involved, the intricate ways in which those parts communicate, and the overall behavior of the system - how do all these parts work together to complete the systems' goals? To ensure the correctness and reliability of such systems, it is of outmost importance that thought is put into the design. After all, failure in a system can lead to negative consequences ranging from minor inconviniences to death[SOURCE Lecture 1].\\
The Industry 4.0 production (I4.0) domain is characterized by the integration of physical components with different kinds of technology and the communication channels between. In this domain there exist many complex production systems, one of which is the production of cars. Cars are sold not only for their functionality as a vehicle of transport, but often offer the customers a range of options for customization. These range from functional (such as the engine) to cosmetic (such as the color). Building a system that can handle customer orders and organize production of the placed orders is no small feat. It is one thing to design an architecture for such a system, but how does one ensure that it fulfills the requirements? Changes to the system during later parts of the development process are much more costly than changes made earlier in the process. Therefore it is important to ensure that the architecture is designed well, before implementation starts. For a system which customers directly interract with, an important aspect is the performance. Not only must the system provide quick feedback to customer actions, it must also be able to do so while servicing multiple customers at the same time. From this, we've created these research questions:
\emph{Research questions:}
\begin{enumerate}
    \item  How to design software architecture for performance?
    \item How to evaluate the software architecture in regards to performance?
    \item How well does a prototype built based on this architecture perform perfomance-wise? 
\end{enumerate}

\emph{Approach.}
The following steps are taken to answer this paper's research questions:
\begin{enumerate}
    \item smth about each section of the report? Like, use cases were made to get an idea of what the system should be able to do?
\end{enumerate}


\section{Related work}
\label{sec:related_work}
% ... Should review the state of the art consisting of 8-10 papers and should contextualize how this study provides new knowledge to the field.
% this is about reconfigurability! we need to focus on performance! 
This Section addresses existing contributions by examining xxx in the I4.0 domain. 
In total, x papers are investigated. 

In \cite{Wan2019Reconfigurable}, experiences are elaborated on a three-layer architecture of a reconfigurable smart factory for drug packing in healthcare I4.0. 

The paper \cite{Yazen2010Ontology} proposes an ontology agent-based architecture for inferring  new configurations to adapt to changes in manufacturing requirements and/or environment.

In \cite{Leitao2016Specification,Angione2017Integration} an architecture for a reconfigurable production system is specified.
Two objectives for reconfiguration and how they can be reached are described.

Several papers \cite{Koren1999Reconfigurable,Koren2010Design,Bortolini2018Reconfigurable} describe reconfigurable manufacturing systems that are cost-effective and responsive to market changes.

All contributions provide valuable knowledge about reconfiguration but lack a study of the software architecture perspective that specifies a quantifiable reconfigurability architectural requirement, a software architecture that adopts the architectural requirements, and evaluates the architectural requirement. 

\section{Use Case}
\label{sec:use_case}
% From lecture 12 Summary-keypoint-exam topics sample: 
% Unfold the problem with a use case and describe what the use case is about
This Section introduces the use case of a customer placing an order on the company's website. This use case covers the sequence of actions starting from the customer accessing the website, to the moment when the order is saved and marked as 'ready' for production in the system.
\subsection{Customer places an order}
Actors: Customer, Website, Production Scheduler
Preconditions: Available cars and customization options have been defined in database.
Steps:
\begin{itemize}
    \item Customer accesses company website
    \item Customer chooses preferred car
    \item Website fetches available customization options for chosen car
    \item Customer chooses preferred customizations
    \item Customer places the order
    \item Website sends order details to the Scheduler
    \item Production Scheduler recieves order details from website
    \item Production Scheduler breaks down the order into a JSON production recipe, which is then stored in the database
    \item Production Scheduler stores the order in the database, marked as 'ready'
\end{itemize}
Postconditions: A customer order is stored and marked as 'ready' for production.

\section{Quality Attribute Scenario}
%lecture 12 - the use case is the foundation to describe and and specify architectural requirements. 
This section contains the quality attributes for the architecture and also contains one or more Quality Attribute Scenarios (QAS), which illustrate the system's scenario.
\subsection{Performance}
\label{sec:qas}

The performance of the system starts from the moment a customer has placed an order on the website, and the performance depends on the website, Non/Non-relational DB, Kafka Message bus, and Production Scheduler. Communication between these systems must take place within ten units.\\

\textbf{Scenarios:} Two scenarios have been made for Performance.\\

The first scenario on 1 shows that when a customer places an order through the car-selling website, it will publish to Kafka that an order has been created. After this event is done, the production scheduler fetches the message and starts generating the JSON production recipe, and marks it as ready as soon as it is stored in the database. This entire event chain will be completed in 10 time units. The customer's order submission to Kafka to get the message; this process must complete within four milliseconds. This scenario ensures that distributed communication does not cause significant latency and captures the system's performance requirement for the major workflow.

\begin{figure}[h]
    \centering
    \includegraphics[width=0.8\linewidth]{quality attribute diagram 1.drawio.png} % adjust filename
    \caption{Scenario 1 for the Performance}
    \label{quality attribute diagram 1.drawio}
\end{figure}

The second scenario on 2 shows that after the Kafka message bus gets the order from the website, the scheduler must start immediately to prevent delays. In normal load conditions, as soon as the message is available in Kafka, the scheduler must start generating the recipe within three milliseconds. This prevents performance bottlenecks in the message queue and also ensures that the event-driven architecture facilitates quick transitions from order creation to recipe development.

\begin{figure}[h]
    \centering
    \includegraphics[width=0.8\linewidth]{quality attribute diagram 2.drawio.png} % adjust filename
    \caption{Scenario 2 for the Performance}
    \label{quality attribute diagram 2.drawio}
\end{figure}

\subsection{Requirements} Based on the QAS and the use case, several requirements were defined, which can be seen in table \ref{ReqTabel}[FIX TABLE LABEL!!]. 
\begin{center}
\begin{tabular}{ | m{0.7cm} | m{4cm}| m{1.5cm} | m{1.7cm} | }
 \hline
 Req ID & Description & Type & QA \\ 
 \hline
 RQ1 & Customer can choose from a list of available cars & Functional & Us\\ 
 \hline
 RQ2 & Customer should be able to chose a car and customize the car's model, engine and color & Functional & qa\\ 
 \hline
 RQ3 & Scheduler should prepare a placed order for production by creating and saving a production recipe & Functional & Perf\\
 \hline
 RQ4 & Scheduler should keep track of an order's status (ready, in progress, completed) & Functional & per\\
 \hline
 RQ5 & The website should respond to user actions in less than 3 seconds & Non-Functional & Performance\\
 \hline
 RQ6 & Any order placed on the website should be saved and marked as 'ready' for production in database within 10 time units & Non-Functional & Performance\\ 
 \hline
\end{tabular}
\label{ReqTabel}
\end{center}


\section{Design and Analysis Modelling}
\label{sec:design_and_analysis_modelling}
This section describes the different steps in the design process, as well as an analysis of the design.
%Describe the design and argue for the design decision and how it meets the QASes. Part of the design decision must specify which tactics/patterns are used (provide arguments) and the trade-offs. 
%Analyze the design and illustrate the system architecture model, encompassing both structure and behavior, using formal reasoning.
% Feature model is optional. However, it should be explained what/which particular features (characteristics) are selected and the reason of choice.
% A list of requirements. Diagram is optional. 
% High level architecture structure: components and their interactions to meet the RQs.
%Offers a clear and more comprehensive design, modeling and analysis with both a detailed description and a well-considered evaluation of pros and cons, including a clearer explanation of how the design and modeling affect Quality Attributes with relevant requirement analysis and tactics.

%TODO AM
% present tactics
% Discuss the trade-offs
% then structural model -> behavioral model

\subsection{Tactics}
% smth about kafka separating website from scheduler, so website doesnt have to wait. Replicas of website?
To ensure the health of the connection from the systems, the most fundamental tactic is event-based processing, which separates the Website from the Scheduler and guarantees that user-facing operations do not hinder generating the JSON production recipe or database writing. It also covers the bound execution times tactics, as after the customer places the order, the message bus will receive the message within three milliseconds, and the Production Scheduler will generate the recipe within three milliseconds. To reduce overall latency, the system additionally depends on concurrency by simultaneously allowing the Scheduler to process messages, create recipes, and communicate with the database. By doing that, it avoids bottlenecks, and concurrent processing guarantees that the Scheduler can process incoming Kafka events.

\subsection{Features}
Figure \ref{FeatureDiagram} shows the features for the car production system that will realize the defined requirements. The system will have a website, from which the customer may browse (RQ1) and choose a car to customize (RQ2). The Scheduler will process the orders by creating a production recipe for each order (RQ3). The Scheduler also manages the status of each order (RQ4), setting status to 'ready' for production when the corresponding production recipe has been created and stored in the system. 
\begin{figure}[h]
    \centering
    \includegraphics[width=0.9\linewidth]{FeatureModel.pdf}
    \caption{Feature Diagram of the car production system}
    \label{FeatureDiagram}
\end{figure}

\subsection{Structural Analysis Model}
A structural analysis model (also called smth FAA) can be used to present the functions for the features defined in figure \ref{FeatureDiagram}, and the way data flows between the different parts. Figure \ref{schedulerFAA} shows the FAA of the Scheduler. 
\begin{figure}[h!]
    \centering
    \includegraphics[width=0.83\linewidth]{SchedulerFAA.jpg}
    \caption{Functional Analysis Model of Scheduler}
    \label{schedulerFAA}
\end{figure}

\subsection{Finite-State Machine}
To show the behavior of the system, finite-state machines (FSM) were defined for the website and scheduler, shown in figure \ref{websiteSM} and \ref{schedulerSM} respectively. 
\begin{figure}[h!]
    \centering
    \includegraphics[width=1\linewidth]{SMWebsite.pdf}
    \caption{Finite State Machine for the website}
    \label{websiteSM}
\end{figure}
When accessing the website, all available products are shown to the user. The user may navigate to the sub-page containing the details for a specific car, where they may customize the car and add it to their cart. Adding it to their cart will redirect them to the cart, where they can checkout to place their order. Finally, they will be shown a confirmation screen. From any page on the website, they directly return to the homepage. 
\begin{figure}[h]
    \centering
    \includegraphics[width=1\linewidth]{SchedulerSM.pdf}
    \caption{Finite State Machine for the scheduler}
    \label{schedulerSM}
\end{figure}
When the scheduler starts, it will listen for messages from the website. Once a message has been received containing details of an order, the scheduler will process the order, before returning to listening. 

\subsection{Technologies}
%not sure where we should talk about actual technologies and tools used. It should deffo be in the report, but not sure it should be here?? i dunno
% perhaps it should be moved to 'experiment design' in 'Evaluation', cuz the technologies are relevant for the experiment, and also this from exercise 10 which was abt experiment: "Decide on the tools, environments, and configurations necessary for the experiment."
This section will outline the design patterns suitable for the project.

 \subsection{Apache Kafka:} To establish communication between the website and the production scheduler, Apache Kafka will be used. The architecture requires a message bus for the subsystems to communicate with each other, and the system does not require a third-party broker or messages to be organized by \ref{https://www.hivemq.com/blog/mqtt-vs-kafka-real-time-bidirectional-data-processing/}"Topics," which is why Kafka is chosen instead of MQTT. Kafka decouples data streams and systems. The website sends data into Kafka, and the production scheduler queries Kafka for the data. It scales very well and will not  require integration with every system.
 

 \subsection{ Systems and subsystems}
 \begin{itemize}
    \item Website
 \begin{itemize}
     \item Backend
      \begin{itemize}
        \item Programming language: Python Django framework
        \begin{itemize}
            \item Comprehensibility of the language makes it easier and faster to write understandable code, and easier to comply with the system. 
            \item It has enough libraries to automate testing, as it can easily conduct testing. Also, it offers a platform that allows automation based on non-browser functionality. 
        \end{itemize}
        \item Database: Relational Database 
    \end{itemize}
    \end{itemize}
\end{itemize}
\begin{itemize}
        \item Fronted 
      \begin{itemize}
        \item Programming language:HTML,CSS,JavaScript 
        \begin{itemize}
            \item Runs anywhere, no installation needed.
            \item UI updates instantly for all users.
        \end{itemize}
    \end{itemize}
\end{itemize}

 \begin{itemize}
    \item Scheduler

      \begin{itemize}
        \item Programming language: Python 
        \item Database: Relational Database(MYSQL) 
        \begin{itemize}
            \item MySQL offers speed and reliability, especially high-speed read operations, where PostgreSQL seems to be more suited for complex queries, data integrity, and consistency. MySQL is better suited for the system, as the system has simple queries and wants speed.
        \end{itemize}
        \item Non-Relational Database(MongoDB)
        \begin{itemize}
            \item MongoDB is a document-oriented database. MongoDB stores data in the form of key-value pairs. MongoDB is high-speed, high-availability, and scalability, and also supports JSON formats and allows for ACID transactions.
        \end{itemize}
    \end{itemize}
    \end{itemize}

\subsection{Containerization}
This subsection will contain the subsystems that are going to be containerized in the project and how they impact the performance attributes of the system. For containerization, Docker is going to be used.
\subsubsection{Website}
The website has been chosen to be containerized as it has to ensure that the execution environment is the same every time, regardless of the underlying operating system. The website must be scalable 
since the quantity of users' orders can change over time.
\subsubsection{Production Scheduler}
The Production Scheduler has been chosen to be containerized as it is responsible for coordinating orders, generating production recipes, and scheduling production runs. It ensures the isolation between the scheduling logic and other parts of the system. It improves fault tolerance.
\subsubsection{Impact}
The impact of Docker Compose provides the system with maintainability and scalability. Containerization improves scalability by enabling the deployment of more instances of the Website or Scheduler as system demand rises. It also improves maintainability as it ensures that the service always runs in the same environment and works the same way on different operating systems.

\section{Formal verification and validation}
\label{sec:formal_v_and_v}
%Offers a comprehensive and in-depth description of the V&V, covering detailed methodology, model checking execution, and thorough analysis of the verification/simulation results (including a table/list of property V&V results) and system/design improvement based on the interpretation of V&V results, demonstrating an advanced understanding of experimental methods in software engineering Formal verification and validation of system(s).

% lecture 8 + 9
In order to perform a formal verification and validation of the architecure, a model is created to represent the system, based on the Finite-State Machine defined in section \ref{sec:design_and_analysis_modelling}. Timing constraints are identified for the events within the model, based on the requirements. Then, the formal model can be verified to see whether it upholds the constraints and requirements, by simulating it in the model-checker tool UPPAAL [cite upaal somehow]. Figure XX shows the model for the website, and figure XX shows the scheduler.\\
The properties to be evaluated, written in the UPPAAL query language, can be seen in table \ref{VVResults}, along with the result from the verification. If a property fails, UPPAAL outputs a counter example (CE) to show how it failed. 
% What about the CTL tree thingy?
\begin{center}
\begin{tabular}{ | m{3cm} | m{1.5cm}| m{2cm} | }
 \hline
 Property & Result & Counter Example \\ 
 \hline
 A[] not deadlock & pass/fail & Ce or N/A \\ 
 \hline
 \( E<>\) state XX reached within time XX... & rgt & gr \\
 \hline
\end{tabular}
\label{VVResults}
\end{center}
Based on these results,...


\section{Evaluation}
\label{sec:evaluation}
% Empirical evaluation - the actual experiment
This Section describes the evaluation of the proposed design.
Section \ref{sec:design} introduces the design of the experiment to evaluate the system. 
Section \ref{sec:measurements} identifies the measurements in the system for the experiment.
Section \ref{sec:pilot_test} describes the pilot test used to compute the number of replication in the actual evaluation. 
Section \ref{sec:analysis} presents the analysis of the results from the experiment. 

\subsection{Experiment design}
\label{sec:design}
The experiment design will be based off the process presented in Claes Wohlin et al. 2012**. Thus, our goal template will be:
\begin{itemize}
    \item Analyze our software architecture
    \item For the purpose of checking whether the requirements are met with respect to the performance QA
    \item From the point of view of the customer and developer
    \item In the context of a customer using the system (website and scheduler) to place an order for a customized car. %the system being locally hosted on a computer, deployed in Docker containers (or physical location? Like classroom? Or an informal setting as compared to a more formal/professional setting like a lab study). (Single-object study OR Multi-object variation study)
\end{itemize}

The scenario we'll be testing is: "From the moment a customer has placed an order on the website, the order must be placed in ‘queue’ (recipe should be saved in DB, and orderStatus should be ‘ready’), within 10 (timeunits**)". 

\subsection{Measurements}
\label{sec:measurements}
The time will be measured in (mili)seconds for an action to finish.

\subsection{Pilot test}
\label{sec:pilot_test}

\subsection{Analysis}
\label{sec:analysis}


\section{Conclusion}
\label{sec:conclusion}
Conclusion of the report, discussion and relevant future work.

\subsection{Discussion}
\label{sec:Discussion}

\subsection{Future work}
\label{sec:future_work}

\bibliographystyle{IEEEtran}
\bibliography{references}
\vspace{12pt}

\newpage
\section*{Contributions}
\begin{tabular}{l|p{0.6\linewidth}}
Name & Contribution\\
Anne-Marie & Scheduler, Kafka, Use case, Design and Analysis Modelling\\
\end{tabular}
\end{document}
